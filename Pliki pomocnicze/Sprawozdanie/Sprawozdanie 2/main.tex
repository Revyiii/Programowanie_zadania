\documentclass[a4paper,11pt]{article}

\usepackage{latexsym}%symbole
\usepackage[empty]{fullpage}
\usepackage{titlesec}%alternatywne sekcje tytułowe
\usepackage{marvosym}%wiecej symboli
\usepackage[usenames,dvipsnames]{color}
\usepackage{verbatim}%wyświetlanie kodu
\usepackage{enumitem}%zmienianie layoutu
\usepackage[hidelinks]{hyperref}%?!?!?!
\usepackage{fancyhdr}%heders and footers
\usepackage[english]{babel}%znaki lokalne?
\usepackage{tabularx}% tabeli 
\usepackage{multicol}% wiele column
\usepackage{minted}% syntax hilighting
\input{glyphtounicode}

\usepackage{baskervillef}
\usepackage[T1]{fontenc}
%flow chatry
\usepackage{tikz}
\usetikzlibrary{shapes.geometric, arrows}
\usepackage{amssymb}
\usepackage{float}

\pagestyle{fancy}
\fancyhf{} 
\fancyfoot{}
\setlength{\footskip}{5pt}
\renewcommand{\headrulewidth}{0pt}
\renewcommand{\footrulewidth}{0pt}

\usepackage[bottom=0.5in,top=0.5in, left=0.5in ,right=0.5in]{geometry}

\urlstyle{same}

%\raggedbottom
\raggedright
\setlength{\tabcolsep}{0in}

\titleformat{\section}{
  \it\vspace{3pt}
}{}{0em}{}[\color{black}\titlerule\vspace{-5pt}]

\pdfgentounicode=1
%flowchatrts shapes

\definecolor{1}{RGB}{128, 147, 241}
\definecolor{2}{RGB}{114, 221, 247}
\definecolor{3}{RGB}{179, 136, 235}
\definecolor{4}{RGB}{247, 174, 248}

\tikzstyle{startstop} = [ellipse, rounded corners, minimum width=2cm, minimum height=1cm,text centered, draw=black,thick,fill=4]
\tikzstyle{io} = [trapezium,trapezium stretches=true,trapezium left angle=70,trapezium right angle=110,thick,minimum width=2cm,minimum height=0.85cm, text centered, draw=black,fill=1]
\tikzstyle{process} = [rectangle,minimum width=3cm,minimum height=0.85cm,text centered,text width=3cm,draw=black,thick,fill=2]
\tikzstyle{decision} = [diamond,minimum width=1cm, minimum height=1cm, text centered, draw=black, fill=3,thick]
\tikzstyle{arrow} = [thick,->,>=stealth]

\begin{document}
\begin{center}
    \begin{multicols}{2}
    \begin{flushleft}
    \large{Tymon Łazowy 3D nr.09} \\
    \end{flushleft}
    
    \begin{flushright}
    \large{29.09.2023}\\
    \end{flushright}
    \end{multicols}
    {\LARGE Sprawozdanie z informatyki nr 2} \\ \vspace{0pt}
\end{center}

\section{Treść zadanie}
\begin{center}
\large{Programownie część Druga}\\

\begin{itemize}[ label={}]
    \normalsize{\item{
    {1.0 - Wyświetl napis powitalny, 2 wiersze poniżej ndk, cls, pożegnanie}{} \\
    {1.1 - Wartość wyrażenia, 2 miejsca po przecinku: $\frac{2*3+17}{9}$ (wynik $2,56$) }{}\\
    {1.2 - Wartość wyrażenia, 3 miejsca po przecinku: $\frac{4^2+2*4*7+7^2}{5+37\div4}$ (wynik $8,491$) }{} \\
    {1.3 - Powitanie, cls, pole pow. i objętość sześcianu o boku 5,875 cm (2mpp) (p=207,09; obj=202,78) \\
    {1.4 - Pole objętość i suma długości krawędzi sześcianu o boku 7.225 cm (ze stałą)
$P=313,20; O=377,15; S=86,70$}{} \\
    {1.5 - Średnia arytmetyczna z 3 liczb - stałe( 7,12,16) oraz napis NDK
Średnia=11,67}{} \\
    {1.6 - Pole, objętość i suma długości krawędzi prostopadłościanu o bokach podanych
przez użytkownika (2mpp) }{} \\
    {1.7 - Pole, objętość i suma długości krawędzi walca o promieniu i wysokości podanych
przez użytkownika (pi jako stała)}{} \\
    {1.8 - Średnia arytmetyczna z 3 liczb podanych przez użytkownika z jego imieniem.}{} \\
    {1.9 - Pole prostokąta, ze sprawdzeniem danych (czy boki są większe od 0)}{} \\
    {1.10 - Czy dwie liczby podane przez użytkownika są podzielne przez siebie – pierwsza
przez drugą – uwaga na warunek podzielności}{} \\
    {1.11 - Równanie liniowe w pełnej postaci $(ax+b=c)$}{}\\
    {1.12 - Równanie $ax^2+bx+c=d$ kwadratowe lub liniowe – wyniki na dole ekranu}{}\\
    {1.13 - Wyświetlanie maksymalnej liczby z trzech podanych liczb całkowitych ze sprawdzeniem poprawności danych}{}\\
    {1.14 - Sprawdzenie możliwości skonstruowania trójkąta z trzech odcinków, których długości są losowymi liczbami rzeczywistymi, losowanymi z przedziału od <1,10> z jednym miejscem po przecinku (wyświetla liczby na górze i komunikat na dole ekranu}{}\\
}}}
\end{itemize}

\end{center}
%---------------------PROPONOWANE ROZWIĄZANIA-------------
%pagebreak 3, 6, 8, 9, 11
\section{Proponowane rozwiązania}
\subsection*{1.0}
%----------------1.0------------------
\begin{multicols}{2}
  \begin{flushleft}
    \begin{figure}[H]
    \centering  
\begin{tikzpicture}[node distance=1.5cm]

\node (start) [startstop] {Start};
\node (out1) [io, below of=start] {Wyświetl powitanie};
\node (out2) [io, below of=out1] {2 linie poniżej ndk};
\node (in1) [io, below of=out2] {poczekaj aż użytkownik ndk};
\node (pro1) [process, below of=in1] {Wyczyść terminal};
\node (out3) [io, below of=pro1] {Pożegnaj się};
\node (stop) [startstop, below of =out3]{stop};

\draw [arrow] (start) -- (out1);
\draw [arrow] (out1) -- (out2);
\draw [arrow] (out2) -- (in1);
\draw [arrow] (in1) -- (pro1);
\draw [arrow] (pro1) -- (out3);
\draw [arrow] (out3) -- (stop);

\end{tikzpicture}
    \caption{1.0 flowchart}
    \label{flow0}
\end{figure}
    \end{flushleft}
D: brak

W: brak
    \begin{flushright}
    \begin{minted}[framesep=1mm, baselinestretch=1.,linenos]{cpp}
    cout << "Witam w programie" << endl<< endl;
    cout << "Nacisnij dowolny kalwisz";
    getch();
    system("cls");
    cout << "do widzenia";
    getch();
\end{minted}
    \end{flushright}
\end{multicols}
\subsection*{1.1}
%-------1.1---------


%-------------alternatywne rozwiązamnia
\section{Alternatywne rozwiązania}
\begin{center}
\large \textbf{python}
\end{center}

%--------------Programy użyte----------
\section{Programy użyte do wykonoania zadań}  
\LaTeX, google chrome, overleaf, dev c++, python, visual studio code, notepad++, git, github, Sumatra PDF, Total comander  
\section{Wnioski i uwagi} 
\begin{center}
 \large{Zadanie mi się bardzo podobało i nie mam żadnych uwag.}   
\end{center}
 
\end{document}
