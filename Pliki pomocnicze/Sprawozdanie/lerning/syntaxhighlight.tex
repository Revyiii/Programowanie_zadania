\documentclass{article}
\usepackage{minted}
\begin{document}
\begin{minted}[frame=lines, framesep=1mm, baselinestretch=1.2,linenos]{cpp}
#include <iostream> // input out put stream
#include <cstdlib>
#include <iomanip>
#include <conio.h>
#include <math.h>
#include <string>
#include <cstring>
#include <Windows.h>
#include <locale.h>//do zawarcia polskicg znaków
using namespace std;

//polskie znaki 
setlocale(LC_CTYPE,"Polish");

//zmienne
int n=0;
char* pow_arr = new char[pow.length()];
float bok = 5.875; // zminna zmieno przecinkowa
string pow ("witam w programie"); //słowa w zmiennej

//funkje
Sleep(2000);//czekaj (windows.h)
delete[] pow_arr;// usuń obiekt
strcpy(pow_arr, pow.c_str());
n++ //zmiena dodać 1  
cout << pow_arr[n]; // wyświetl na konsoli
system("cls");//czyszczenie konsoli
cout << setprecision(5) << "p="<< pole << "; obj=" << obj; //wyświetl zmienną z precyzją do n liczb po przecinky
return 0; // zwróć zero (kniec)
cin.get();// czeka na enter #iostream
getch(); // ndk #conio.h

//pętle
while(n<=pow.length()){} //pętla 

// funkje, definicje
int main(){}
\end{minted}
%\inputminted[frame=lines, framesep=1mm, baselinestretch=1.2,linenos,]{cpp}{szyfrownik.cpp}




\end{document}