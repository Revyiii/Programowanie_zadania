\documentclass[12pt,]{article}


\usepackage{baskervillef}%polskie znaki może nie

%flow chatry
\usepackage{tikz}
\usetikzlibrary{shapes.geometric, arrows}
\usepackage{amssymb}
\usepackage{float}

\title{Sprawozadanie lernicng}
\author{Tymon Łazowy}
\date{123,2123,12}


\tikzstyle{startstop} = [ellipse, rounded corners, minimum width=2cm, minimum height=1cm,text centered, draw=black,thick,fill=red!0]

\tikzstyle{io} = [trapezium,
trapezium stretches=true,
trapezium left angle=70,
trapezium right angle=110,
thick,minimum width=2cm,
minimum height=0.85cm,
text centered,
draw=black,
fill=blue!0]

\tikzstyle{process} = [rectangle,minimum height=0.85cm,text centered,draw=black,thick,]

\tikzstyle{decision} = [diamond,minimum width=1cm, minimum height=1cm, text centered, draw=black, fill=green!0,thick]

\tikzstyle{arrow} = [thick,->,>=stealth]

\begin{document}
D: brak

S: możliwość zbydownia trójkąta \{tak, nie\} 
\begin{figure}[h]
   
    \caption{1.14 flowchart} 
    \centering  
\begin{tikzpicture}[node distance=1.5cm]

\node (start) [startstop] {Start};
\node (pro1) [process, below of=start,text width=2.5cm, yshift=-0.2cm,text width=4cm,] {zaok.dół(a,b,c random $\in \mathbb{R}$<0,1>*100)/10 };
\node (pro2) [process, below of=pro1,text width=2.5cm, yshift=-0.2cm] {posortój a,b,c rosąco};
\node (dec1) [decision, below of=pro2, yshift=-1.cm] {a+b>c};
\node (tak1) [io, left of=dec1, xshift=-0.8cm, yshift=-1.5cm] {można trójkąt};
\node (nie1) [io, right of=dec1, xshift=0.8cm, yshift=-1.5cm] {nie można};
\node (meet) [coordinate,below of =dec1,yshift=-1cm]{};
\node (stop) [startstop, below of =meet]{stop};


\draw [arrow] (start) -- (pro1);
\draw [arrow] (pro1) -- (pro2);
\draw [arrow] (pro2) -- (dec1);
\draw[arrow] (dec1) -| (tak1) node[pos=0.25,fill=white,inner sep=3]{Tak};
\draw[arrow] (dec1) -| (nie1) node[pos=0.25,fill=white,inner sep=3]{Nie};   
\draw [arrow] (tak1)|-(meet);
\draw [arrow] (nie1)|-(meet);
\draw [arrow] (meet)--(stop);



\end{tikzpicture}
    
    \label{flow14}
\end{figure}
\end{document}